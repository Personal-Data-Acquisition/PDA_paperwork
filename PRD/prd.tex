\hypertarget{product-requirements-documentprd}{%
\section{Product Requirements
Document(PRD)}\label{product-requirements-documentprd}}

\textbf{Authors:} Jake Goodwin, Aidan Agee, Blake Babb, Patrick Iacob

\textbf{DATE:} 2023-11-23

\hypertarget{problem-description}{%
\subsection{Problem Description}\label{problem-description}}

Existing personal data acquisition devices are either too expensive or
too DIY for most potential users. Many are designed for aerospace,
automotive or research purposes, and are too expensive and unnecessarily
complicated for casual users. The only other option is for users to
build their own devices from prefabricated parts, which is too
complicated and requires too much much prerequisite knowledge for most
potential users.

\hypertarget{scope}{%
\subsubsection{Scope}\label{scope}}

The scope of this project is to develop a prototype for a personal data
acquisition device. This prototype will have the ability to collect real
time data from a variety of sensors, including accelerometers,
gyroscopes, GPS modules, and thermometers. These components will need to
combined in a printed circuit board for the final prototype. The scope
of this project also includes development of a web-based UI that both
presents the data gathered by the prototype and sends commands to the
physical device to record data and configure sensors.

\hypertarget{use-cases}{%
\subsubsection{Use Cases}\label{use-cases}}

The user will take the product along with them on an outdoor activity
and subject it to normal conditions for that activity.

The user will connect the product to a phone or laptop they brought with
them, and view the data stream and take samples using the user
interface, and save the data locally.

The user will choose and connect selected modules to the system using
the CAN(controller area network) bus.

\hypertarget{purpose-and-visionbackground}{%
\subsection{Purpose and
Vision(Background)}\label{purpose-and-visionbackground}}

Our purpose is to develop a personal data acquisition system that
records all the data a user might want, and is cheap and easy to set up
and use. It should be able to record data on acceleration, force,
position, etc. require minimal setup, and can be hooked up to bike,
go-kart, etc.

\hypertarget{stakeholders}{%
\subsection{Stakeholders}\label{stakeholders}}

\textbf{Capstone Team}\\
The capstone team are the main decision makers for the project, and will
need extensive information for the product's requirements and
implementation details. They will also need oversight from the project
partner and TA.

\textbf{Project Partner}\\
The project partner will be working very closely with the capstone team,
and will need to know the teams capabilities and status, and the status
of the project.

\textbf{Project TA}\\
The TA needs to be informed on project progress and any issues the team
may be having.

\textbf{Capstone Instructors}\\
The instructors require much of the same information as the TA, but
because they are working less closely with the team there is less
urgency.

\textbf{Users}\\
Users will need to know the product's capabilities, limitations and
intended use.

\hypertarget{preliminary-context}{%
\subsection{Preliminary Context}\label{preliminary-context}}

\hypertarget{assumptions}{%
\subsubsection{Assumptions}\label{assumptions}}

\begin{itemize}
\tightlist
\item
  We have a suitable power supply of 12v to power the system.
\item
  The end user has a device capable of connecting to an ad-hoc network.
\item
  The data to be logged doesn't require more speed than the CAN 2.0
  standard.
\item
  The environment it's meant to be used in is electrically noisy.
\end{itemize}

\hypertarget{constraints}{%
\subsubsection{Constraints}\label{constraints}}

\begin{itemize}
\tightlist
\item
  As undergraduate students, our team has limited experience in the
  field, so we will have to learn a lot to deliver the product.
\item
  Our budget is limited, so we will have to choose components carefully
  based on price.
\item
  We are limited to three terms to deliver our product.
\end{itemize}

With these constraints factored, the biggest concerns for the
feasibility of our project are the skills that need to be learned and
limited time alloted to do so and complete the project. As an example,
the team has primarily non-formal experience in hardware organization
but has thus far worked efficiently in that aspect of the project. These
risks are mitigated by the expertise and technical support offered by
our project partner, and we consequently find the scope of our project
realistic.

\hypertarget{dependencies}{%
\subsubsection{Dependencies}\label{dependencies}}

\begin{itemize}
\tightlist
\item
  The rust language, (reduces bugs and helps with memory safety.)
\item
  C compiler(s), (C ABI is still used as a way to interface with libs.)
\item
  Rust Embassy Library. (Embedded rust lib to reduce boilerplate)
\item
  Rust Rocket(web server)
\item
  STM SDK and HAL (Good refences for the actual hardware.)
\item
  The CAN standard.
\item
  The Unix networking stack
\item
  SQLite and or rust file I/O
\item
  Rust Libraries available for individual sensor modules.
\end{itemize}

Some possible bottlenecks that could occur given our current
dependencies would be centered around sensor modules not having an
existing library written in rust. This would add more development time
to the project.

However, we've researched workarounds and discovered tools to generate
the needed interfaces for rust from a C header file.

\hypertarget{market-assessment-and-competition-analysis}{%
\subsection{Market Assessment and Competition
Analysis}\label{market-assessment-and-competition-analysis}}

\textbf{RexGen:} Proprietary CAN bus based data logger, hard to find
tutorial or documentation and is prohibitively expensive for hobbyists.
Also unable to guarantee that their system is memory safe.

\textbf{CANedge1:} CANedge1: It has open source elements to it and
documentation that is accessible, but still does not meet the
requirements for its cost.

DEWEsoft sells test and measurement equipment. Their products are not a
good fit for our users because they are designed for industry, and
therefore overkill and are prohibitively expensive for an individual.

Omega Engineering sells data loggers that can record the data our users
would want, can connect to a remote device over Bluetooth and have easy
to use interfaces. However most of their data loggers only record one or
two types of data, so a user would need to buy many of them, which would
be inconvenient and expensive.

An Apple Watch can track a user's activity data, and send it to an
iPhone with an easy to read interface. However, the Apple Watch is
limited in what kind of data it can record, and would not be appropriate
for our users due to its many other unneeded functions.

There are guides on the internet that instruct a user on how to build
their own data acquisition device using Arduino or Raspberry Pi
microcontroller much more inexpensively than the other alternatives.
However, this requires the user to have background knowledge in
circuitry and programming, and requires a lot of time and effort to set
up.

\hypertarget{target-demographics-user-persona}{%
\subsection{Target Demographics (User
Persona)}\label{target-demographics-user-persona}}

Terry is an amateur Go-kart enthusiast who was brought into the hobby 8
months ago by friends and has become entrenched in the hobby since then.
They are looking for a way to improve their performance but need more
information about their current racing habits to do that.

Alice is a CTO of a large company that has decided to data log the
forces and location their products experience during shipping through
multiple contracted pilots and routes. She needs a system that isn't
cost prohibitive to deploy in large numbers and can be customized for
her company's other projects as needed.

John is an extreme snowboarder looking to collect data from his downhill
tricks in order to help his friend create realistic and smooth
animations for a snowboarding video game. He needs a data logging system
that can endure cold environments and is modular so he can keep down the
bulk/weight of the system while carving toeside and hitting some sweet
jumps.

James is a competition mountain biker who wants to record and analyze
data during rides for performance improvement. Uses a smartphone and
needs an easy-to-use interface. He needs a system to compare data
between runs.

\hypertarget{requirements}{%
\subsection{Requirements}\label{requirements}}

\hypertarget{user-stories-and-features-functional-requirements}{%
\subsubsection{User Stories and Features (Functional
Requirements)}\label{user-stories-and-features-functional-requirements}}

\begin{longtable}[]{@{}lllll@{}}
\toprule
\begin{minipage}[b]{0.37\columnwidth}\raggedright
User Story\strut
\end{minipage} & \begin{minipage}[b]{0.18\columnwidth}\raggedright
Feature\strut
\end{minipage} & \begin{minipage}[b]{0.08\columnwidth}\raggedright
Priority\strut
\end{minipage} & \begin{minipage}[b]{0.09\columnwidth}\raggedright
GitHub Issue\strut
\end{minipage} & \begin{minipage}[b]{0.14\columnwidth}\raggedright
Dependencies\strut
\end{minipage}\tabularnewline
\midrule
\endhead
\begin{minipage}[t]{0.37\columnwidth}\raggedright
As a mountain biker, I want to be able to view my instantaneous speed at
any point in my journey.\strut
\end{minipage} & \begin{minipage}[t]{0.18\columnwidth}\raggedright
Gps\strut
\end{minipage} & \begin{minipage}[t]{0.08\columnwidth}\raggedright
Must Have\strut
\end{minipage} & \begin{minipage}[t]{0.09\columnwidth}\raggedright
TBD\strut
\end{minipage} & \begin{minipage}[t]{0.14\columnwidth}\raggedright
Common firmware.\strut
\end{minipage}\tabularnewline
\begin{minipage}[t]{0.37\columnwidth}\raggedright
As a motorsport hobbyist, I want to be able to record the g-forces I
experience while going around tight corners.\strut
\end{minipage} & \begin{minipage}[t]{0.18\columnwidth}\raggedright
Accelerometer.\strut
\end{minipage} & \begin{minipage}[t]{0.08\columnwidth}\raggedright
Must Have\strut
\end{minipage} & \begin{minipage}[t]{0.09\columnwidth}\raggedright
TBD\strut
\end{minipage} & \begin{minipage}[t]{0.14\columnwidth}\raggedright
Common firmware.\strut
\end{minipage}\tabularnewline
\begin{minipage}[t]{0.37\columnwidth}\raggedright
As a winter sports enthusiast, I want to be able to track the turning
speed of my snowboard.\strut
\end{minipage} & \begin{minipage}[t]{0.18\columnwidth}\raggedright
Yaw rate sensor.\strut
\end{minipage} & \begin{minipage}[t]{0.08\columnwidth}\raggedright
Must Have\strut
\end{minipage} & \begin{minipage}[t]{0.09\columnwidth}\raggedright
TBD\strut
\end{minipage} & \begin{minipage}[t]{0.14\columnwidth}\raggedright
Common firmware.\strut
\end{minipage}\tabularnewline
\bottomrule
\end{longtable}

\hypertarget{non-functional-requirements}{%
\subsubsection{Non-Functional
Requirements}\label{non-functional-requirements}}

\begin{itemize}
\tightlist
\item
  Delay on data transmission should be at an acceptable level
\item
  Code should be well documented, following coding standards and best
  practices
\item
  User interface should intuitive and fast to use
\item
  The product should use security best practices whenever possible
\end{itemize}

\hypertarget{data-requirements}{%
\subsubsection{Data requirements}\label{data-requirements}}

\begin{itemize}
\tightlist
\item
  Analog data will be converted to digital.
\item
  Sensor data must be reliable, resistant to EMI
\item
  Sensor modules must adhere to the CAN protocol.
\end{itemize}

\hypertarget{integration-requirements}{%
\subsubsection{Integration
requirements}\label{integration-requirements}}

\begin{itemize}
\tightlist
\item
  All interfaces will be rust doc documented.
\item
  Tests will ensure API usage integrity.
\item
  All modules that are to use the interface must pass integration tests.
\end{itemize}

\hypertarget{user-interaction-and-design}{%
\subsubsection{User interaction and
design}\label{user-interaction-and-design}}

\begin{itemize}
\tightlist
\item
  Use a web server to interface with the user
\item
  Can display live data, configure sensors, and download data
\item
  Build using EGUI and Rust Rocket
\item
  Should focus on ease of use for less experienced users
\item
  Should be able to clearly provide all data and provide configurability
\end{itemize}

\includegraphics{https://github.com/Personal-Data-Acquisition/PDA_paperwork/assets/114545996/c9bfb6fa-6805-4da0-a16f-07caaa345572}\\
\emph{UI Mockup from project partner}

\hypertarget{user-documentation}{%
\subsubsection{User Documentation}\label{user-documentation}}

The user documentation will be produced in markdown or LaTeX into a PDF
or webpage. This documentation will cover the basic usage of the system
and instructions on how to build it.

\hypertarget{testing-and-quality-assurance}{%
\subsubsection{Testing and Quality
Assurance}\label{testing-and-quality-assurance}}

Testing will be done through TDD(test driven development) using the
supported testing frameworks for rust and C. These tests will allow us
as developers to ensure the assumptions we make about our code matches
the actual behavior of it.

Quality assurance will mostly be handled by adhearance to style
standards enforced by the lanuages LSP(language server protocol)
servers. The two that will see extensive use in this project being:

\begin{enumerate}
\def\labelenumi{\arabic{enumi}.}
\tightlist
\item
  Rust-analyzer
\item
  clangd
\end{enumerate}

Bug and issue tracking will all be handled by GitHub's Issue and project
system. This also serves as a way to allow public contributions in the
future to the code base.

\hypertarget{milestones-and-timeline}{%
\subsection{Milestones and Timeline}\label{milestones-and-timeline}}

\begin{longtable}[]{@{}lll@{}}
\toprule
Item & Description & Duration\tabularnewline
\midrule
\endhead
Schematics & The wiring schematics & 2 months\tabularnewline
PCB & PCB gerber files & 1 month\tabularnewline
uC & Firmware for STM &\tabularnewline
Sensor FW & Sensor module firmware &\tabularnewline
UI & Web user interface &\tabularnewline
Server & The back end web server &\tabularnewline
\bottomrule
\end{longtable}

\hypertarget{open-questions}{%
\subsection{Open questions}\label{open-questions}}

Currently, there are no open questions in the project.

\hypertarget{out-of-scope}{%
\subsection{Out of scope}\label{out-of-scope}}

\begin{itemize}
\tightlist
\item
  Support for more than the listed sensors.
\item
  HAL development
\item
  Radiation Hardening
\item
  Full EMI sheilding
\item
  Full support for 10 channels saturated with sensor data at 5kHz
\item
  Water resistance at any depth or submersion
\item
  Documentation beyond rustdocs/doxygen and markdown/latex.
\item
  Wireless connectivity beyond ad-hoc wifi.
\end{itemize}
